\documentclass[11pt,a4paper,onecolumn]{tiet-question-paper}
\date{28 May 2024}
\institute{Alpha}
\instlogo{images/tiet-logo.pdf}
\schoolordepartment{%
  Computer Science \& Engineering Department}
\examname{%
  End Semester Examination}
\coursecode{UCS505}
\coursename{Computer Graphics}
\timeduration{3 hours}
\maxmarks{45}
\faculty{ANG,AMK,HPS,YDS,RGB}

\begin{document}
\maketitle
\textbf{Instructions:}
\begin{enumerate}
\item Attempt any 5 questions;
\item Attempt all the subparts of a question at one
  place.
\end{enumerate}
\bvrhrule[0.4pt]
\begin{enumerate}
\item
  \begin{enumerate}
  \item Given the control polygon
    $\textbf{b}_0, \textbf{b}_1, \textbf{b}_2,
    \textbf{b}_3$ of a Cubic Bezier curve; determine
    the coordinates for parameter values
    $\forall t\in T$. \hfill [7 marks]

    \begin{align*}
      T \equiv
      & \{0, 0.15, 0.35, 0.5, 0.65, 0.85, 1\} \\
      \begin{bmatrix}
        \textbf{b}_0 &\textbf{b}_1& \textbf{b}_2& \textbf{b}_3
      \end{bmatrix} \equiv
      & \begin{bmatrix}
        1&2&4&3\\ 1&3&3&1
      \end{bmatrix}
    \end{align*}
  \item Explain the role of convex hull in curves.
    \hfill[2 marks]
  \end{enumerate}
\end{enumerate}
\bvrhrule[0.4pt]
\begin{enumerate}[resume]
\item
  \begin{enumerate}
  \item Describe the continuity conditions for
    curvilinear geometry.
    \hfill[5 marks]
  \item Define formally, a B-Spline curve. \hfill [2
    marks]
  \item How is a Bezier curve different from a B-Spline
    curve?
  \end{enumerate}
\end{enumerate}
\bvrhrule[0.4pt]
\begin{enumerate}[resume]
\item
  \begin{enumerate}
  \item Given a triangle, with vertices defined by
    column vectors of $P$; find its vertices after
    reflection across XZ plane. \hfill [3 marks]
    \begin{align*}
      P\equiv
      &\begin{bmatrix}
        3&6&5 \\ 4&4&6 \\ 1&2&3
      \end{bmatrix}
    \end{align*}
  \item Given a pyramid with vertices defined by the
    column vectors of $P$, and an axis of rotation $A$
    with direction $\textbf{v}$ and passing through
    $\textbf{p}$.  Find the coordinates of the vertices
    after rotation about $A$ by an angle of
    $\theta=\pi/4$.\hfill [6 marks]
    \begin{align*}
      P\equiv
      &\begin{bmatrix}
        0&1&0&0 \\ 0&0&1&0 \\0&0&0&1
      \end{bmatrix} \\
      \begin{bmatrix}
        \mathbf{v} & \mathbf{p}
      \end{bmatrix}\equiv
      &\begin{bmatrix}
        0&0 \\1&1\\1&0
      \end{bmatrix}
    \end{align*}
  \end{enumerate}
\end{enumerate}
\bvrhrule[0.4pt]
\begin{enumerate}[resume]
\item
  \begin{enumerate}
    \item Explain the two winding number rules for
      inside outside tests. \hfill [4 marks]
    \item Explain the working principle of a
      CRT. \hfill [5 marks]
  \end{enumerate}
\end{enumerate}
\bvrhrule[0.4pt]
\begin{enumerate}[resume]
\item
  \begin{enumerate}
  \item Given a projection plane $P$ defined by normal
    $\textbf{n}$ and a reference point $\textbf{a}$;
    and the centre of projection as $\mathbf{p}_0$;
    find the perspective projection of the point
    $\textbf{x}$ on $P$. \hfill [5 marks]
    \begin{align*}
      \begin{bmatrix}
        \mathbf{a}&\mathbf{n}&\mathbf{p}_0&\mathbf{x}
      \end{bmatrix}\equiv
      &
        \begin{bmatrix}
          3&-1&1&8\\4&2&1&10\\5&-1&3&6
        \end{bmatrix}
    \end{align*}
  \item Given a geometry $G$, which is a standard unit
    cube scaled uniformly by half and viewed through a
    Cavelier projection bearing $\theta=\pi/4$
    wrt. $X$-axis. \hfill [2 marks]
  \item Given a view coordinate system (VCS) with
    origin at $\textbf{p}_v$ and euler angles ZYX
    $\boldsymbol{\theta}$ wrt. world coordinate system
    (WCS); find the location $\mathbf{x}_v$ in VCS,
    corresponding to the point $\textbf{x}_w$ in
    WCS. \hfill [2 marks]
    \begin{align*}
      \begin{bmatrix}
        \mathbf{p}_v & \boldsymbol{\theta} & \mathbf{x}_w
      \end{bmatrix}\equiv
      &\begin{bmatrix}
        5&\pi/3&10\\5&0&10\\0&0&0
      \end{bmatrix}
    \end{align*}
  \end{enumerate}
\end{enumerate}
\bvrhrule[0.4pt]
\begin{enumerate}[resume]
\item
  \begin{enumerate}
    \item Describe the visible surface detection
      problem in about 25 words. \hfill [1 mark]
    \item To render a scene with $N$ polygons into a
      display with height $H$; what are the space and
      time complexities respectively of a typical
      image-space method. \hfill [2 marks]
    \item Given a 3D space bounded within
      $[0\quad0\quad0]$ and $[7\quad7\quad-7]$,
      containing two infinite planes each defined by 3
      incident points
      $\mathbf{a}_0, \mathbf{a}_1, \mathbf{a}_2$ and
      $\mathbf{b}_0, \mathbf{b}_1, \mathbf{b}_2$
      respectively bearing colours (RGB) as
      $\mathbf{c}_a$ and $\textbf{c}_b$ respectively.
      \begin{align*}
        \begin{bmatrix}
          \mathbf{a}_0&\mathbf{a}_1&\mathbf{a}_2
          &\mathbf{b}_0&\mathbf{b}_1&\mathbf{b}_2
          &\mathbf{c}_a&\mathbf{c}_b
        \end{bmatrix}\equiv
        &\begin{bmatrix}
          1&6&1&6&1&6&1&0 \\
          1&3&6&6&3&1&0&0 \\
          -1&-6&-1&-1&-6&-1&0&1
        \end{bmatrix}
      \end{align*}

      Compute and/ or determine using the depth-buffer
      method, the colour at pixel $\mathbf{x}=(2,4)$ on
      a display resolved into $7\times7$ pixels. The
      projection plane is at $Z=0$, looking at
      $-Z$. \hfill [6 marks]
  \end{enumerate}
\end{enumerate}
\bvrhrule[0.4pt]
\end{document}
